% Para adicionar uma imagem ou incluir um arquivo .tex você precisa 
% adicionar \CWD
% caminho relativo (ao documento principal) do diretório.
% 
% Exemplo:
% \begin{figure}
%   \includegraphics{\CWD/imagens/exemplo.pdf}
% \end{figure}

The main observation that we have to make in order to solve this problem is that every game that Alice and Bob play can be interpreted as a classic Nim game, where a token placed in column $y$ represents a stone pile of size $y - 1$.

To determine the winner of a Nim game, we have to calculate the nim-sum of the game, which is $a_1 \oplus a_2 \oplus a_3 \cdots \oplus a_n$, where $\oplus$ stands for the XOR logical operator and $a_1, a_2, \cdots, a_n$ are the sizes of the stone piles of the game. Alice wins if nim-sum is different from $0$, otherwise Bob wins.

To solve the problem efficiently, we have to process games in non decreasing order of its value $r$. We maintain a variable $nimsum$, which holds the nim-sum value of games up to the current column $r$ we are analizing. Every time we find a token whose column $y$ lies in the range $[1, r]$ we make $nimsum = nimsum \oplus (y - 1)$. After that we just have to determine if $nimsum$ equals $0$ or not in order to find out the winner.

We have to carefully print the answers of every game in the order given in the input.

The time complexity of this approach is $O(c \log c)$.