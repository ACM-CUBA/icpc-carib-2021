The first step to solve this problem is to determine when a solution exists.
For this, it's helpful to analyse the problem as a graph, where cats create
directed edges to dogs and viceversa.

In particular the following happens:

\begin{itemize}

    \item each cat will add $a$ edges, and each dog will add $b$ edges, so in
        any solution there will be $n \cdot a + n \cdot b$ edges.

    \item the maximum amount of edges that can be created in any solution, such
        that there is not a cat and a dog looking each other simultaneously, is
        $n \cdot n$.

\end{itemize}

So, $n \cdot a + n \cdot b \le n \cdot n$ holds. Simplifying we get that $a + b
\le n$. This is a necessary condition for the existence of a solution.

With this in mind, let's create a construction to place the edges on the graph.
The idea is: for every cat create $a$ edges to the following $a$ dogs
cyclically (starting from 0 if we are on the last dog), more formally: the
$i$-th cat will add edges to dogs $i \cdot a \mod n, (i \cdot a + 1) \mod n,
\ldots, (i \cdot a + a - 1) \mod n$ (all indexes start from 0).

To complete the construction, each dog $j$ will add exactly $b$ edges to cats
$i$ such that the edge $(i, j)$ was not added on the step explained below. It
turns out that each dog will be able to find the needed $b$ cats.

To see this, note that the maximum in-degree of a dog needs to be at most $a$,
if not, there were more than $n \cdot a$ edges added from cats. So, it follows
that the amount of available cats for each dog is at least $n - a$, and by the
inequality shown above $n - a \ge b$ (each dog will have enough cats to add
edges to them).
